\documentclass[a4paper, 12pt]{article}
\usepackage[polish]{babel}
\usepackage[utf8]{inputenc}
\usepackage[T1]{fontenc}
\usepackage{graphicx}
\usepackage{float}
\usepackage{hyperref}
\usepackage{geometry}
\geometry{a4paper, margin=2cm}

\title{Protokół z ćwiczeń cz. II: Oszacowanie obciążenia genetycznego}
\author{Mikołaj Mieszko Charchuta}
\date{\today}

\begin{document}

\maketitle

\section{Wprowadzenie}
Celem badania było oszacowanie obciążenia genetycznego u dwóch blisko spokrewnionych gatunków ptaków: biegusa łyżkodziobego (*Calidris pygmaea*) oraz biegusa rdzawoszyjego (*Calidris ruficollis*). Biegus łyżkodzioby jest gatunkiem krytycznie zagrożonym wyginięciem, co czyni go idealnym obiektem do badania wpływu małej liczebności populacji na erozję genetyczną.

\section{Materiały i Metody}
Do analizy wykorzystano dane sekwencjonowania genomowego dwóch osobników: biegusa łyżkodziobego (C\_pyg\_26) oraz biegusa rdzawoszyjego (C\_ruf\_09). Dane ograniczono do analizy scaffoldu 1 genomu referencyjnego biegusa łyżkodziobego.

\subsection{Przygotowanie danych}
Sekwencje scaffoldu 1 wyodrębniono z genomu referencyjnego za pomocą narzędzia \textbf{seqkit}. Plik FASTA scaffoldu 1 zindeksowano przy użyciu \textbf{samtools faidx}.

\subsection{Mapowanie odczytów}
Odczyty zmapowano do genomu referencyjnego za pomocą \textbf{bwa mem}. Pliki BAM posortowano (\textbf{samtools sort}) i usunięto duplikaty (\textbf{picard MarkDuplicates}).

\section{Wyniki}
\subsection{Analiza jakości odczytów}
Jakość odczytów przed i po filtrowaniu była wysoka, choć wykryto regiony o niskiej jakości, prawdopodobnie związane z błędami sekwencjonowania.

\subsection{Identyfikacja SNP}
Dla osobnika C\_pyg\_26 wykryto 242 SNP po filtrowaniu (z 244504 wariantów początkowych). Dla osobnika C\_ruf\_09 wykryto 1314 SNP po filtrowaniu (z 240385 wariantów początkowych).

\section{Dyskusja}
Wyniki wskazują, że biegus rdzawoszyi ma wyższe obciążenie genetyczne w porównaniu do biegusa łyżkodziobego, co może wynikać z większej liczby homozygotycznych wariantów.

\end{document}